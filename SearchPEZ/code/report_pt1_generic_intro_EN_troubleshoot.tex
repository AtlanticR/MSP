% Options for packages loaded elsewhere
\PassOptionsToPackage{unicode}{hyperref}
\PassOptionsToPackage{hyphens}{url}
%
\documentclass[
]{article}
\usepackage{lmodern}
\usepackage{amssymb,amsmath}
\usepackage{ifxetex,ifluatex}
\ifnum 0\ifxetex 1\fi\ifluatex 1\fi=0 % if pdftex
  \usepackage[T1]{fontenc}
  \usepackage[utf8]{inputenc}
  \usepackage{textcomp} % provide euro and other symbols
\else % if luatex or xetex
  \usepackage{unicode-math}
  \defaultfontfeatures{Scale=MatchLowercase}
  \defaultfontfeatures[\rmfamily]{Ligatures=TeX,Scale=1}
\fi
% Use upquote if available, for straight quotes in verbatim environments
\IfFileExists{upquote.sty}{\usepackage{upquote}}{}
\IfFileExists{microtype.sty}{% use microtype if available
  \usepackage[]{microtype}
  \UseMicrotypeSet[protrusion]{basicmath} % disable protrusion for tt fonts
}{}
\makeatletter
\@ifundefined{KOMAClassName}{% if non-KOMA class
  \IfFileExists{parskip.sty}{%
    \usepackage{parskip}
  }{% else
    \setlength{\parindent}{0pt}
    \setlength{\parskip}{6pt plus 2pt minus 1pt}}
}{% if KOMA class
  \KOMAoptions{parskip=half}}
\makeatother
\usepackage{xcolor}
\IfFileExists{xurl.sty}{\usepackage{xurl}}{} % add URL line breaks if available
\IfFileExists{bookmark.sty}{\usepackage{bookmark}}{\usepackage{hyperref}}
\hypersetup{
  pdftitle={Reproducible Report for Species in Maritimes Region - for Internal DFO use only},
  pdfauthor={Synthesis prepared by the Reproducible Reporting Team, steering committee and advisors.},
  hidelinks,
  pdfcreator={LaTeX via pandoc}}
\urlstyle{same} % disable monospaced font for URLs
\usepackage[margin=1in]{geometry}
\usepackage{graphicx,grffile}
\makeatletter
\def\maxwidth{\ifdim\Gin@nat@width>\linewidth\linewidth\else\Gin@nat@width\fi}
\def\maxheight{\ifdim\Gin@nat@height>\textheight\textheight\else\Gin@nat@height\fi}
\makeatother
% Scale images if necessary, so that they will not overflow the page
% margins by default, and it is still possible to overwrite the defaults
% using explicit options in \includegraphics[width, height, ...]{}
\setkeys{Gin}{width=\maxwidth,height=\maxheight,keepaspectratio}
% Set default figure placement to htbp
\makeatletter
\def\fps@figure{htbp}
\makeatother
\setlength{\emergencystretch}{3em} % prevent overfull lines
\providecommand{\tightlist}{%
  \setlength{\itemsep}{0pt}\setlength{\parskip}{0pt}}
\setcounter{secnumdepth}{-\maxdimen} % remove section numbering
\usepackage{booktabs}
\usepackage{longtable}
\usepackage{array}
\usepackage{multirow}
\usepackage{wrapfig}
\usepackage{float}
\usepackage{colortbl}
\usepackage{pdflscape}
\usepackage{tabu}
\usepackage{threeparttable}
\usepackage{threeparttablex}
\usepackage[normalem]{ulem}
\usepackage{makecell}
\usepackage{xcolor}

\title{Reproducible Report for Species in Maritimes Region - for Internal DFO
use only}
\author{Synthesis prepared by the Reproducible Reporting Team, steering
committee and advisors.}
\date{23 February, 2021}

\begin{document}
\maketitle

{
\setcounter{tocdepth}{2}
\tableofcontents
}
\hypertarget{reproducible-reporting}{%
\section{\texorpdfstring{\textbf{REPRODUCIBLE
REPORTING}}{REPRODUCIBLE REPORTING}}\label{reproducible-reporting}}

\hypertarget{about-this-report}{%
\subsubsection{\texorpdfstring{\textbf{About this
report}}{About this report}}\label{about-this-report}}

This document summarizes regional data, including information on species
and habitats, available at the time of report creation. \ldots{}

\hypertarget{how-to-use-this-report}{%
\subsubsection{\texorpdfstring{\textbf{How to use this
report?}}{How to use this report?}}\label{how-to-use-this-report}}

\begin{itemize}
\tightlist
\item
  This document
\end{itemize}

\newpage

\hypertarget{search-results}{%
\subsection{\texorpdfstring{\textbf{SEARCH
RESULTS:}}{SEARCH RESULTS:}}\label{search-results}}

\hypertarget{emphasis-on-species-listed-by-the-species-at-risk-act-assessed-by-cosewic-andor-wild-species-table-x}{%
\subsection{\texorpdfstring{\textbf{EMPHASIS ON SPECIES LISTED BY THE
SPECIES AT RISK ACT, ASSESSED BY COSEWIC AND/OR WILD SPECIES (Table
x)}}{EMPHASIS ON SPECIES LISTED BY THE SPECIES AT RISK ACT, ASSESSED BY COSEWIC AND/OR WILD SPECIES (Table x)}}\label{emphasis-on-species-listed-by-the-species-at-risk-act-assessed-by-cosewic-andor-wild-species-table-x}}

\begin{table}

\caption{\label{tab:unnamed-chunk-2}Definitions relevant for species listed by *Species At Risk Act* (SARA), assessed by the Committee on the Status of Endangered Wildlife in Canada (COSEWIC), or included in the Wild Species listings.}
\begin{tabular}[t]{l|l}
\hline
Document / Organization & Description\\
\hline
SARA & The Species at Risk Act (SARA) was proclaimed in June 2003, and is one part of a three part Government of Canada strategy for the protection of wildlife species at risk. This three part strategy also includes commitments under the Accord for the Protection of Species at Risk and activities under the Habitat Stewardship Program for Species at Risk. The purposes of SARA are to prevent Canadian indigenous species, subspecies, and distinct populations from becoming extirpated or extinct, to provide for the recovery of endangered or threatened species, and encourage the management of other species to prevent them from becoming at risk. It applies to all federal lands in Canada; all wildlife species listed as being at risk; and their critical habitat. For more information on SARA visit <https://www.canada.ca/en/environment-climate-change/services/species-risk-public-registry.html>\\
\hline
COSEWIC & The Committee on the Status of Endangered Wildlife in Canada (COSEWIC) is an independent advisory panel to the Minister of Environment and Climate Change Canada that meets twice a year to assess the status of wildlife species at risk of extinction. Members are wildlife biology experts from academia, government, non-governmental organizations and the private sector responsible for designating wildlife species in danger of disappearing from Canada. COSEWIC determines the national status of wild Canadian species, subspecies, varieties or other designatable units that are suspected of being at risk of extinction or extirpation. COSEWIC uses a process based on science and Aboriginal or community knowledge to assess wildlife species at risk. All native mammals, birds, reptiles, amphibians, fish, arthropods, molluscs, vascular plants, mosses and lichens are included in COSEWIC's current mandate.
For more information on COSEWIC visit <https://www.cosewic.ca/index.php/en-ca/>\\
\hline
Wild Species & Every five years the National General Status Working Group (NGSWG), which includes representatives from all provincial and territorial governments in Canada, and from the federal government, publishes a general report on the status of wildlife species in Canada. 
The following text is verbatim from the Wildlife website (URL below).These reports are meant to inform Canadians about the status of species in the country, and to help prevent species in Canada from becoming extinct as a consequence of human activity.
For more information about Wild Species reports visit <https://www.wildspecies.ca/home>\\
\hline
\end{tabular}
\end{table}

\end{document}
